\chapter*{Abstract} %*-Variante sorgt dafür, das Abstract nicht im Inhaltsverzeichnis auftaucht

Das DLRG Dienstplan-Portal dient zur Vereinfachung der Planung von Diensten im Ehrenamt. Der Prozess des Ausschreibens zu besetzender Dienste, der Helfer-Meldung wie auch jährlichen Statistik werden mit diesem Portal abgebildet.

\vspace*{5mm} \noindent Ziel ist es den ehrenamtlichen Helferinnen und Helfern eine Übersicht des aktuellsten Standes der Diensteinteilung wie auch selber Einflussmöglichkeiten zu geben. Die Helfer können selbst  partizipieren und Dienstwünsche direkt einreichen. 

\vspace*{5mm} \noindent Das Benutzerhandbuch für das DLRG Dienstplan-Portal hat den Zweck, eine Übersicht der Funktionen aus Sicht eines Anwenders zu geben. In diesem Dokument sind die einzelnen Funktionen und Prozesse im Detail erläutert und mit Grafiken als auch Beispielen aus der Anwendung verdeutlicht. \\
Genderhinweis: Aus Gründen der besseren Lesbarkeit wird auf eine geschlechtsneutrale Differenzierung verzichtet. Entsprechende Begriffe gelten im Sinne der Gleichbehandlung grundsätzlich für beide Geschlechter. Die verkürzte Sprachform beinhaltet keinerlei Wertung.