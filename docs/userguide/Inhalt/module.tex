\chapter{Module}
\label{cha:module}
Die Applikation ist in modularen Modulen untergliedert. Je nach Anforderung und Bedürfnis können einzelne Module für Gliederungen aktiviert werden.
Die aktuell aktiven Module werden den Administratoren der Gliederung unter der Gliederungsverwaltung \noindent (Abbildung \ref{fig:view_client} \textit{\nameref{fig:view_client}}, Markierung \textit{1}, rechts oben) aufgelistet\\

\noindent Die aktuell aktivierbaren Module sind folgend aufgelistet:

\begin{itemize}
	\item[\textbf{Dienstplan:}] Dies stellt das Kernmodul dar welches automatisch für jede Gliederung aktiv ist. Das Modul beinhaltet die Grundfunktionen wie Benutzerverwaltung, Qualifikationen, Nachrichten, E-Mail Versand wie auch die Dienstplanung selbst.

	\item[\textbf{Fortbildungen:}] Mit diesem Modul werden Fortbildungen verwaltet. 
	In Abgrenzung zu Diensten können Positionen einer Fortbildung durch beliebig viele Benutzer besetzt werden. 
	Modul-Voraussetzungen: Dienstplan
	
	%\item[\textbf{Analysen:}]
\end{itemize}
