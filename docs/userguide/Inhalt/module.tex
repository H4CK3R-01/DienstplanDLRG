\chapter{Module}
\label{cha:module}
Die Applikation ist in modularen Modulen untergliedert. Je nach Anforderung und Bedürfnis können einzelne Module für Gliederungen aktiviert werden.
Die aktuell aktiven Module werden den Administratoren der Gliederung unter der Gliederungsverwaltung \noindent (Abbildung \ref{fig:view_client} \textit{\nameref{fig:view_client}}, Markierung \textit{1}, rechts oben) aufgelistet\\

\noindent Die aktuell aktivierbaren Module sind folgend aufgelistet:

\begin{description}
	\item[Dienstplan:] Dies stellt das Kernmodul dar welches automatisch für jede Gliederung aktiv ist. Das Modul beinhaltet die Grundfunktionen wie Benutzerverwaltung, Qualifikationen, Nachrichten, E-Mail Versand wie auch die Dienstplanung selbst.

    \item[Abfragen und Bestätigungen:] Aktiviert das Verwalten von Abfragen und Bestätigungen dieser. Abfragen dienen zur Belehrung und Bestätigung und dokumentieren Rückmeldungen der Nutzer. Z.B. Durchführung von UVV Unterweisung oder Abfragen von aktuellen Kontaktdaten.

	\item[Fortbildungen:] Mit diesem Modul werden Fortbildungen verwaltet.
	In Abgrenzung zu Diensten können Positionen einer Fortbildung durch beliebig viele Benutzer besetzt werden. 
	Modul-Voraussetzungen: Dienstplan

    \item[Credits für Fortbildungen:] Aufbauend auf dem Modul Fortbildungen können Credits für einzelne Fortbildungspositionen verwaltet werden. Hiermit sind Fortbildungsübersichten über Saisons für jeden User in seinem Profil einsehbar.

    \item[Erweiterte Statistik:] Aktiviert das Modul Statistik welches die Möglichkeit der Auswertung von Zeiträumen sowie erweiterte Reports ermöglicht.

\end{description}
