
% alle Abkürzungen, die in der Bachelorarbeit verwendet werden

\nomenclature{CI}{Corporate Identity }

\nomenclature{Dienst}{Eine Freiwilligenarbeit für eine definierte Zeit. Meist im Wasserrettungsdienst oder Freibad. Der Dienst definiert wann und eventuell wie lange und wo die Freiwilligearbeit zu tätigen ist. Ein Dienst enthält eine oder mehrere zu besetzende Positionen.}

\nomenclature{Gliederung}{Repräsentiert eine hierarchische Organisations-Einheiten (z.B. Bundesverband, Landesverband, Bezirke oder Ortsgruppe). Typischerweise wird in der Softwareentwicklung equivalent die Begriffe Mandant oder Client verwendet.}

\nomenclature{GUI}{Grafische Benutzeroberfläche oder auch grafische Benutzerschnittstelle (graphical user interface) }

\nomenclature{ORM}{Objektrelationale Abbildung (object-relational mapping) }

\nomenclature{Position}{Eine zu besetzende Freiwillgearbeit mit vordefinierten vorraussetzungen wie Wissen und Können (Qualifikation). Eine Position ist immer einem Dienst zugeordnet.}

\nomenclature{Qualifikation}{Ein definiertes Wissen und Können welches meist durch Besuchen von Lehrgängen und Fortbildungen erlangt werden kann.}

\nomenclature{RWD}{(Responsive Web Design) Die Größe und Auflösung der Displays auf Laptops, Desktop-PCs, Tablets und Smartphones können erheblich variieren. Aus diesem Grund ist das Erscheinungsbild und die Bedienung einer Website stark abhängig vom Endgerät. Aufgrund dessen werden Webseiten mit einem Design ausgestattet welches sich auf die unterschiedlichen Anforderungen der Endgeräte anpassen. \cite{Responsive_Webdesign}
}